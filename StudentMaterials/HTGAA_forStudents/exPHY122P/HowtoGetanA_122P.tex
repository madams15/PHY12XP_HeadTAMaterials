\documentclass[12pt]{article}

\linespread{1}

\usepackage[english]{babel}
\usepackage{color}
\usepackage{enumerate}
\usepackage{fancyhdr}
\usepackage{float}
\usepackage[T1]{fontenc}
\usepackage{fullpage}
\usepackage[textheight = 675pt]{geometry}
\usepackage[pdftex]{graphicx}
\usepackage{hyperref} \hypersetup{colorlinks=true, urlcolor=blue}
\usepackage{ifpdf}
\usepackage[options]{pdfpages}
\usepackage{wrapfigure}
\usepackage{yfonts}

\pagestyle{fancy}
\setlength{\headheight}{10pt}
\lhead{Marissa Adams *}
\chead{\emph{University of Rochester}}
\rhead{PHY122P, Fall 2015}}
\lfoot{* e-mail: \href{mailto:madams@pas.rochester.edu}{madams@pas.rochester.edu} }

\makeatletter
\renewcommand*{\p@section}{\S\,}
\renewcommand*{\p@subsection}{\S\,}
\renewcommand*{\p@subsubsection}{\S\,}
\makeatother

\begin{document}
\hspace{1pt}

\begin{center}
\textbf{\Large How to get an A in Mastery/Self-Paced Physics}
\end{center}

\noindent If you want the quick run down, see the end of the document for \ref{sec:tldr} (the too long; didn't read). In this document we tell you how to get an A in PHY122P. To get an A you need to familiarize yourself with:

\begin{enumerate}
	\item \ref{sec:cons} The Constraints of the Course, which involves,
	\item \ref{sec:gs} the Grade Composition, the \ref{sec:p} People available to you, when \ref{sec:w} Workshops are, what to expect for a \ref{sec:pre} Prescreen, and \ref{sec:q} Quizzes.
	\item And \ref{sec:oe} Our Expectations. Some of these are elaborated on through \ref{sec:a} Analogies.
\end{enumerate}

\noindent You should also ask yourself what \ref{sec:ye} Your Expectations will be for the semester.

\section{Constraints of the Course} \label{sec:cons}

Here is a discussion, or rather a list, of what aspects of the course are \textbf{not} likely to change. These are facts that you will have to work with. However e-mail the Head TA if you have any comments or concerns.

\subsection{Grade Composition} \label{sec:gs}

\noindent Three major components to the course: Workshops, Labs, and a Final.

\begin{itemize}
	\item Four subcomponents to workshops: \textcolor{magenta}{workshop problem modules}, \textcolor{magenta}{prescreening}, \textcolor{cyan}{module quizzes (59\%)}, and \textcolor{cyan}{attendance (6\%)}.
	\item Two subcomponents to labs: \textcolor{cyan}{pre-labs (5\%)}, and \textcolor{cyan}{post-labs (5\%)}.
	\item There is one \textcolor{cyan}{Final (25\%)}.
\end{itemize}

\noindent Notice that the components typed in \textcolor{cyan}{cyan} contribute to your grade calculation. The components typed in \textcolor{magenta}{magenta} do not make up a percentage of your grade. 

\subsection{People \& Resources} \label{sec:pr}

\subsubsection{People} \label{sec:p}

\begin{itemize}
	\item PHY122P is staffed by two Professors (Professors Arie Bodek, and Steven Manly), one Head TA, 5 graduate TAs, and 6 senior self-paced veteran undergraduate TIs. \emph{Thus we have 14 skilled physics aficionados helping you learn physics}.
	\begin{itemize}
		\item There will always be 1-2 TA/Is working B\&L 208, where workshops are held.
		\item These people have three roles: Prescreener, Workshop Leader, and Grader.
	\end{itemize}
	\item There are currently around 165 students enrolled in PHY122P. \emph{Thus we have 164 other people learning physics, who can also help you learn physics}.
	\item For outside help, there is *FREE* physics tutoring in the Physics, Optics, and Astronomy (POA) Library, on the third floor of Bausch \& Lomb Hall, every MTWR from 7pm-9pm.
	\item If you would like to pay for a tutor, contact the Undergraduate Coordinator (Janet Fogg (\href{mailto:janf@pas.rochester.edu}{janf@pas.rochester.edu})), or the Graduate Coordinator (Laura Blumkin (\href{mailto:laurablumkin@pas.rochester.edu}{laurablumkin@pas.rochester.edu})). They will let you know if any undergraduate, or graduate students are looking to tutor.
\end{itemize}

\setlength{\headheight}{15pt}

\subsubsection{Workshops} \label{sec:w}

There are 12 module topics (see Syllabus on Blackboard) that you will need to conquer this semester, which is 15 weeks long (not including finals week). Discussions of these topics will occur in a workshop setting.

\begin{itemize}
	\item There are 14 workshops running every week (M, T, W, R, F), except holidays (see Syllabus on Blackboard).
	\begin{itemize}
		\item Monday-Thursday: 14:00-16:40, 16:40-19:20, and 19:20-22:00.
		\item Friday: 14:00-16:40, 16:40-19:20 (no night workshop).
	\end{itemize}
	\item There are exam slots every day but Friday.
	\begin{itemize}
		\item Monday \& Tuesday both have 43 exam slots.
		\item Wednesday \& both have 86 exam slots. 
		\item There are 258 exam slots per week.
		\item It should take you 1.5 tries to pass an exam, $1.5*164 = 246$. Therefore the number of exam slots is reasonable.
	\end{itemize}
	\item You can sign up for exams three different ways (see Introduction to the Course presentation on Blackboard).
\end{itemize}

\subsubsection{Prescreens} \label{sec:pre}

Once you complete your module workshop problems, you are probably ready to take your quiz. Before you take your quiz, we want to see if you have a good understanding of what was involved in the module.

\begin{itemize}
	\item Bring your module problems, or your PHY122P Portfolio to the prescreener.
	\item The prescreener will be sitting in the back of B\&L 208 near the computers.
	\item The prescreener will look through your module workshop problems.
	\item The prescreener may ask you questions on the problem module, they will also ask you to solve one of the problems on the module workshop without reference.
	\item If you have consistently gotten less than a 70\% after two module quizzes, your prescreen will be different. You will have to complete a new problem set.
\end{itemize}

\subsubsection{Quizzes} \label{sec:q}

Quizzes will illustrate your mastery of a subject. There are 12 module quizzes (see Syllabus on Blackboard).

\begin{itemize}
	\item I shouldn't have to tell you this: If you are found cheating, or copying/stealing the quizzes there will be consequences.
	\item We are watching you.
	\item You are not allowed to bring anything into the exam room. We will give you a pencil, calculator, and scrap paper.
\end{itemize}

\subsubsection{Resources} \label{sec:r}

\begin{itemize}
	\item We have a dedicated classroom area: Bausch \& Lomb Hall, room 208. There, we run the workshops. In the adjoining rooms, you will take quizzes, and have them graded. B\&L 208 is located down the left side of the hall (entering from the Eastman Quadrangle) on the second floor.
	\item There are textbooks provided in B\&L 208, so you don't necessarily need to bring your own (unless you want to, of course).
	\item The textbook for the course, \emph{Fundamentals of Physics, 8/10th Edition} by Halliday, Resnick, and Walker (HRW) is available in the POA Library reserves.
	\item \emph{Physics: Principles with Applications} by Giancoli, which might be a nice supplement to your reading or problem solving, is also available in the POA Library (on reserve, and also for general use in the library for after-hours -- see near the ``Free Book'' Shelf (\textbf{note:} They are not allowed to leave the library, but the reserves can for up to 2 hours)).
	\item Video lectures from the lecture-based PHY122 (by Professor Gao) will be posted on Blackboard throughout the semester.
	\item Professor Manly will post his slides \& audio of pertinent lectures from when he taught PHY114.
	\item Powerpoint slides from when Professor Watson taught PHY122 will be posted.
	\item Ultimately there will be resources provided by 1-3 different professors that you are allowed to use in this class. Figure out what works for you!
\end{itemize}

\section{Our Expectations} \label{sec:oe}

\begin{itemize}
	\item The components of the course that are \textcolor{magenta}{magenta} impact the components that are \textcolor{cyan}{cyan}, which contribute to your grade. This means you should take the \textcolor{magenta}{magenta} components very, very seriously.
	\item You will print out your module workshops problems (located on Blackboard) and bring them to workshop (we do not physically supply these for you).
	\item You will buy a binder, or some sort of folder/organizer for this class. There you will store your module workshop problems, your work on them, notes, and feedback forms. We call this your PHY122P Portfolio. We will ask to see it during prescreening. Keep it organized.
	\item You will not lose your feedback forms. We need proof that you passed your quizzes, in case of future paperwork failures. Help us help you.
	\item You will attend your weekly workshops, and be on time. If you are not on time, your attendance will not be recorded. See \ref{sec:gs} for how \textcolor{cyan}{attendance} factors into your grade.
	\item In workshops you will get to know your TA/I, as well as your fellow classmates.
	\item You will work in groups of no more than 6 (B\&L 208 is not a social lounge). You are your partners will learn how to solve physics problems together. 
	\item You will not blatantly copy off of your partners/classmates. This does not help you on the quiz. It is tempting to get the solution quickly. However in this class you have an environment where there is no pressure to do that. Note that in \ref{sec:gs} the workshop problem modules are noted in \textcolor{magenta}{magenta}. They do not contribute to your grade. However they implicitly do through your quiz grade, which illustrates your mastery. Focus on learning how to \emph{master} electricity and magnetism problems at your own \emph{pace}.
	\item The TA/I is not there to solve the problems for you. They are there to help facilitate your learning. The TA/I giving you the solution does not help you learn, even if they are working it out for you in real time. In physics you learn by doing.
	\item This should not discourage you from asking for help. The TA/I can teach you the art of problem solving, or clarify on a physics concept. Asking questions is always a good thing. However the question should not involve the answer being the TA/I solves the problem for you directly.
	\item If you cannot attend workshop (due to illness, religious reasons, etc), you will let both the TA/Is of your assigned and walk-in workshops know ahead of time. Also ensure that they have marked you down appropriately for your walk-in workshop (so that it counts toward your attendance). You are welcome to attend one of the night workshops to make up your attendance.
	\item You can come in to whatever workshop you want, any time, to prescreen. However if you have questions, save them, and come into the beginning of the following workshop, or visit our friends in the POA Library. You are there to prescreen, not do workshop, so do not distract the TA/I running the workshop. That TA/I needs to focus on the students in their workshop.
	\item It is highly encouraged to come to multiple workshops each week. In fact you can do all of your work for the course in our workshops. Just make sure you are on time. However after there are 20 students in the workshop, the workshop leader will need to turn students away. If you are one of those students, you can come to the following workshop, or go to free tutoring.
	\item You will not save up quizzes for the end of the semester.
	\item We will not be doing any grade-a-thons, or making extra quiz slots. You will work with what we give you. Plan accordingly.
	\item You will utilize any of the three different ways you can sign up for quizzes (refer to the Introduction to the Class presentation on Blackboard).
	\item You will not juggle module workshops. Why bother doing problems in module 3 if you have not yet passed module 2??
	\item When taking a module quiz, you will try your best to make your answer as clear, and readable as possible. In order to get 90\% or higher (passing) we need to understand your solution! You will be given scrap paper to use, so save the module quiz sheet for after you've worked out the problems on the scratch paper.
	\item Many engineering students loOOOOoove numbers. However this is a physics course, not an engineering course. Yes, we will give you problems with numbers, but if you want to solve the problem efficiently save the numbers for the end calculation. Do not plug them in immediately until you have a symbolic result of what you're looking for. Using variables is like eating vegetables, they are good for you -- and will make you a better engineer.
	\item Workshop is not math-help time. If you are having trouble with the mathematics involved in the course, please e-mail the Head TA, or the Instructors, and we will point you to the right resources to help you. We are here to help you learn physics, not teach you how to factor polynomials, or understand what a logarithm is.
\end{itemize}

\setlength{\headheight}{15pt}

\subsection{Some Philosophy}

When you hear the word ``physicist,'' you may think of great names, like Albert Einstein, Richard Feynman, or Isaac Newton. These men are associated with this air of ``genius.'' Many people don't think of themselves as geniuses, so when it comes to physics, they don't think it is for them. For these people they might find physics to be impossible, or even scary, and intimidating.

These feelings are natural due to how society perceives people who do physics, but they are also false. Anyone can do physics if they really want to, but it does take a significant amount of time and effort to study it. This is because learning physics does not just involve listening to someone talk about it, it involves ``doing.'' 

Doing things, like solving problems, can take a lot of time, especially when you first begin. However if you keep persisting through the problems, and the initial fears of when you are confronted with a new problem, it will start to get easier. You will notice that some problems are similar, and you'll start to develop categories for different types of problems. 

It is like deciding you are going to run a 5k everyday even though you have never ran before. The first few days may suck, but after a month, you will be smooth sailing: you'll learn to conquer steep hills, and sprint on straight-aways. Learning physics by ``doing'' has been shown to work, which is why we have implemented this course, which facilitates this type of learning, at the University of Rochester.

\section{Your Expectations} \label{sec:ye}

\noindent If you want to get an A, your expectations will need to meet ours. See \ref{sec:oe}.

\section{Analogies} \label{sec:a}

\begin{itemize}
	\item \emph{Think of ``mastery'' in the context of a video game}. Each module workshop is like a level. Each module quiz is like a boss at the end of the level. If you cannot pass the boss at the end of the level, or pass your module quiz, you start the level over, or you need to revisit topics in the module. This system implies that you cannot reach level 2 unless you have successfully completed level 1. Similarly, do not focus on module 2, if you have not passed the module 1 quiz. Then the final is the big scary boss at the end. 
\end{itemize}
	
\noindent \textbf{Moral of the story:} You cannot fully understand Amp\`{e}re's Law, and the Biot-Savart Law (Module 8), if you have not yet passed the module 7 quiz  (on magnetism), etc. The order is there for a reason. Do not bother studying modules in tandem.

\begin{itemize}
	\item \emph{Think of ``self-paced'' in the context of drinking orange juice}. You want to win an orange juice drinking competition, which requires you to drink 12 pints of orange juice. Similarly you want to get an A in this course, which has 12 modules. Would you rather drink 12 pints of orange juice at your own pace throughout the semester, or would you want to drink 12 pints of orange juice on the last day of the semester?
\end{itemize}
	
\noindent \textbf{Moral of the story:} Do not save up module quizzes or prescreens for the end of the semester. 

\section{tl;dr} \label{sec:tldr}

\noindent No, it is not that long. You have the time. Go back and read this document.

\end{document}