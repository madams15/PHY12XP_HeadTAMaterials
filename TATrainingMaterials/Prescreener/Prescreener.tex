\documentclass[11pt]{article}

\linespread{1}

\usepackage[english]{babel}
\usepackage{color}
\usepackage{enumerate}
\usepackage{fancyhdr}
\usepackage{float}
\usepackage[T1]{fontenc}
\usepackage{fullpage}
\usepackage[textheight = 675pt]{geometry}
\usepackage[pdftex]{graphicx}
\usepackage{hyperref} \hypersetup{colorlinks=true, urlcolor=blue}
\usepackage{ifpdf}
\usepackage[options]{pdfpages}
\usepackage{wrapfigure}
\usepackage{yfonts}

\pagestyle{fancy}
\setlength{\headheight}{10pt}
\lhead{M. Adams, A. Bodek *}
\chead{\emph{University of Rochester}}
\rhead{PHY122P, Fall 2015}}
\lfoot{* e-mail(s): \href{mailto:madams@pas.rochester.edu}{madams@pas.rochester.edu},
\newline \href{mailto:bodek@pas.rochester.edu}{bodek@pas.rochester.edu} }

\makeatletter
\renewcommand*{\p@section}{\S\,}
\renewcommand*{\p@subsection}{\S\,}
\renewcommand*{\p@subsubsection}{\S\,}
\makeatother

\begin{document}
\hspace{1pt}

\begin{center}
\textbf{\Large Prescreener}
\end{center}

The purpose of the Prescreen is to determine whether the student is ready to take the module exam. The key question to ask in these instances is whether the student has grasped the key concept of the subject, and whether their mathematical ability is up to the challenge of the exam. The whole point of these Prescreens is to limit the number of the students who just take an exam without really studying. As Prescreener, it is your job to determine if the student is ready to take the module exam.

\begin{itemize}
	\item Prescreens take place in B\&L 208, in the back of the room near a set of computers. You will need a key to this room.
	\item Prescreens only take place during the scheduled workshops. There are 14 workshops running every week (M, T, W, R, F), except holidays (see Syllabus on Blackboard).
	\begin{itemize}
		\item Monday-Thursday: 14:00-16:40, 16:40-19:20, and 19:20-22:00.
		\item Friday: 14:00-16:40, 16:40-19:20 (no night workshop).
	\end{itemize}
	\item Textbooks will be provided in B\&L 208.
	\item Look at the workshop problems for the suggested pace before the week begins. This way you will be prepared.
	\item When a student asks you for a Prescreen, they will need to present to you their \textbf{PHY122P Portfolio}, which will include their worked out solutions to the workshop module problems. How to prescreen:
	\begin{enumerate}
		\item Check their workshop module.
			\begin{itemize}
				\item Make sure they have completed all of the problems.
				\item Browse through the problem solutions to see if the students grasped the concepts.
			\end{itemize}
		\item Talk to the student.
			\begin{itemize}
				\item Prescreens can also involve a short conversation or interview about a type of problem, or concept to delve into the student's understanding.
				\item I conceptual question from the textbook may work well, or explain one of the problems from the workshop module problems.
			\end{itemize}
		\item Have the student redo a problem from the module workshop on scrap paper.
			\begin{itemize}
				\item Students who have not passed an exam after two tries, both of which result in a grading of under 70\%, they will need to complete a new problem set.
				\item Use the textbook, and choose 3-5 problems of varying degrees of difficulty on the topic.
				\item When this student returns after completing the problems, check to see if their answers make sense. 
			\end{itemize}
		\item If the student does not do well during their Prescreen, fill out a Feedback Form.
	\end{enumerate}
	\item When you deem a student prepared to sign up for a module quiz (see Table 1):
	\setlength{\headheight}{15pt}
	\begin{enumerate}
		\item Use the computers provided to update their status on the course grade sheet (``Grade Center'' > ``Full Grade Center'' in Blackboard).
		\item For each student there are columns ``Test \# R-ABCD'' and ``Test \#.'' The following pattern is repeated for all modules in the course.
			\begin{itemize}
				\item The first column is intended to reflect whether a student is ready to take a test, and if he has taken a test which one.
				\item The second column reflects his highest grade on that test.
			\end{itemize}
		\item After the Prescreen, it is important that in the former you put an ``R'' for the appropriate module to reflect the student is ready to take the test. Also note that a student can prescreen for multiple modules without having taken the previous exam module(s); however a student cannot take exam module(s) without having completed the previous ones. When you schedule a student for a test, put the letter S after the R. This way, students are prevented from scheduling multiple tests.
	\end{enumerate}
	\item After the secondary workshop leader is done performing all the Prescreens they can inquire as to whether there are any students who have Feedback Forms they need help with. If there aren't any, they can start to work assist the groups with the work they have going on and the normal flow of the workshop can resume. It is encouraged that the workshop leaders start to rely on each other if they are stuck or unable to explain an idea to a student. Often, a different point of view is needed to explain an idea in such a way that it resonates with a student; this isn�t something to feel bad about. It is also OK to let a student know that you don�t have an answer right away, but will get back to them.
	\item On \textbf{Feedback Forms}:
	\begin{itemize}
		\item These forms are provided after a Prescreen/Exam. On these you should fill out the name of the student in question, the date, and what Prescreen/Exam Module the comments pertain to.
		\item Comments can involve what mistakes they made, and what your recommendation is for them.
		\item Unlike grading, where you give the student a Feedback Form whether or not they passed, when prescreening, you only give a Feedback Form if they did not pass the Prescreen.
		\item The information on this Feedback Form is to be geared for two purposes. The first is for the student who wishes to correct their mistake on their own. The second is, to leave enough of a clue as to the issue so that a workshop leader can assist them. This Feedback Form is an important aspect of the feedback loop between the student and the grader
	\end{itemize}
	\item As a prescreener, you need to be aware that there are three ways students can sign up for exam slots:
	\begin{enumerate}
		\item The first method (\textbf{Just in Time Scheduling}) is straightforward. If a student observes there is an open spot on the exam schedule they can just walk in and take an exam without any formal scheduling. Because of this allowance it is imperative that the grader check the students have passed the Prescreen (and add an S after the R to show that an exam has been scheduled) before administering the exam.
		\item The first method (\textbf{Just in Time Scheduling}) is straightforward. If a student observes there is an open spot on the exam schedule they can just walk in and take an exam without any formal scheduling. Because of this allowance it is imperative that the grader check the students have passed the Prescreen (and add an S after the R to show that an exam has been scheduled) before administering the exam.
		\item If they do not wish to schedule an exam after the Prescreen they can try their luck with the Just in Time Scheduling or send an email and use the last method; \textbf{Remote Scheduling}. If they opt to send the head TA an email, we will need their full name, student id number, the unit number for the exam they wish to take, and their top three preferences for when to take the exam.  Once we schedule a test, the R-ABCD column is changed from R to RS.
	\end{enumerate}
\end{itemize}

\begin{table}[h]
\centering
\label{table:table1}
\begin{tabular}{|p{3cm}|p{3cm}|p{8.5cm}|}
\hline
\textbf{Symbol} & \textbf{Best Grade} & \textbf{Explanation} \\                                                                                               \hline
R & 0 & R = Student has passed the Prescreen for Quiz \#1. \\ \hline
R-S & 0 & Student has passed the Prescreen interview for Quiz \#1 (R), and was scheduled (S) to take the quiz. \\ \hline
R-SAg & 0 & g = Student is taking quiz \#1A, g is removed when student gives the test to the grader. \\ \hline
R-SA & 70 & g is removed after test \#1A is graded. \\ \hline
R-SN & 0 & Student did not show (N) for first appointment to take Quiz \#1, which results in a zero, and still counts as a first attempt. \\ \hline
R-SA-S & 70 & Student was schedule (S) to take Quiz \#1 a second time. \\
R-SA-SBg & 70 & Student is taking quiz \#1B. \\ \hline
R-SA-SB & 100 & g is removed after test \#1B is taken, and the student has passed with more than 80\%. \\ \hline
R-SA-SB & 80 & g is removed after test \#1B is taken with grade of 80. Since student failed two times, they need to pass a more rigorous Prescreen. \\ \hline
R-SA-SB-R & 80 & Student has been prescreened a second time (R2). \\ \hline
R-SA-SB-R-S & 80 & Student has been prescreened a second time (R2), and scheduled to take the test a third time. \\
\hline
\end{tabular}}
\caption{The first two columns illustrate what information will be input into the column labeled ``Test \# R-ABCD'' on Blackboard. Note that 'g' is used to ensure that no test leaves the grading area. The letter 'S' is to prevent students from scheduling Module Exam \#2 before passing Module Exam \#1, or scheduling two slots for the exam.}
\end{table}


\setlength{\headheight}{15pt}

\end{document}