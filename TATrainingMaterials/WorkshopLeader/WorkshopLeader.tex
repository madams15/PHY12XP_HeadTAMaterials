\documentclass[11pt]{article}

\linespread{1}

\usepackage[english]{babel}
\usepackage{color}
\usepackage{enumerate}
\usepackage{fancyhdr}
\usepackage{float}
\usepackage[T1]{fontenc}
\usepackage{fullpage}
\usepackage[textheight = 660pt]{geometry}
\usepackage[pdftex]{graphicx}
\usepackage{hyperref} \hypersetup{colorlinks=true, urlcolor=blue}
\usepackage{ifpdf}
\usepackage[options]{pdfpages}
\usepackage{wrapfigure}
\usepackage{yfonts}

\pagestyle{fancy}
\setlength{\headheight}{10pt}
\lhead{M. Adams, A. Bodek *}
\chead{\emph{University of Rochester}}
\rhead{PHY122P, Fall 2015}}
\lfoot{* e-mail(s): \href{mailto:madams@pas.rochester.edu}{madams@pas.rochester.edu},
\newline \href{mailto:bodek@pas.rochester.edu}{bodek@pas.rochester.edu} }

\makeatletter
\renewcommand*{\p@section}{\S\,}
\renewcommand*{\p@subsection}{\S\,}
\renewcommand*{\p@subsubsection}{\S\,}
\makeatother

\begin{document}
\hspace{1pt}

\begin{center}
\textbf{\Large Workshop Leader}
\end{center}

There are 12 module topics (see Syllabus on Blackboard). The semester is 15 weeks long (not including finals week). Discussions of these topics occur in a workshop setting. You help to facilitate the learning of these topics as workshop leader.

\begin{itemize}
	\item Workshops take place in B\&L 208. You will need a key to this room.
	\item There are 14 workshops running every week (M, T, W, R, F), except holidays (see Syllabus on Blackboard).
	\begin{itemize}
		\item Monday-Thursday: 14:00-16:40, 16:40-19:20, and 19:20-22:00.
		\item Friday: 14:00-16:40, 16:40-19:20 (no night workshop).
	\end{itemize}
	\item Textbooks will be provided in B\&L 208.
	\item Look at the workshop problems for the suggested pace before the week begins. This way you will be prepared.
	\item Each workshop will have a number of enrolled students and some more that just happen to drift in.  \textbf{Students who are enrolled in a section get credit for workshop attendance.} It is encouraged that any student attends as many workshops as they need, but not at your expense, as you need to provide quality support and education for everyone who is assigned to that workshop.
	\begin{itemize}
		\item Groups of no more than six students. The trapezoid shaped tables can only accommodate that many students, anyway.

		\item Get to know the students in your workshop. As time goes on, taking attendance, and tailoring needs, will become easier.
		\item Students who come to attend workshop should be on time, and work in groups. When students arrive, they are matched in reasonably sized groups according to the module that they are working on. 
		\item If the count hits \textbf{twenty students} you should tell students to come at another time, giving students who are enrolled in that section first preference to stay.  However, if a student attends his/her assigned workshop and there are no other students to form a group, then the student should be allowed to work alone.
		\item There are extra spaces for the walk-ins to sign in on the attendance sheet.
		\item Some students may have missed workshop for health or religious reasons, and will attend night workshops to make up for attendance. Make sure that they get their credit. 
	\end{itemize}
	\item Students who want to complete their workshop, work alone, or are primarily there for prescreening should stay. However keep your focus on those who you have taken note of on the attendance sheet.
	\begin{itemize}
		\item Students who are waiting for prescreening should not ask the TA/Is for help.
		\item Refer students who are waiting for Prescreen/working alone, but are asking for help, to the tutoring resources provided by the Society of Physics Students (SPS). The SPS runs a free tutoring program from 19:00-21:00 in the POA Library.  
		\item B\&L 208 is a workshop room and therefore relies on peer learning.  TAs help when the entire group needs help. B\&L 208 should not disintegrate into an individual student help room. If this occurs, pair students together, and enforce group learning.
	\end{itemize}
	\item In the beginning of workshop:
	\begin{enumerate}
		\item To deal with this spread in module level, it is important that at the start of the workshop you ask the students what workshop module they are working on, and break students up into those groups. There are module signs available in the room to label tables, and help you develop a workflow.
		\item Take a poll as to which students wish to perform a Prescreen for any of the exam modules. Refer them to the prescreening area (back of the room, table that is a rectangle). Then the Prescreener deals with them.
		\item Take attendance for the students who are registered for the specific section, so that they will get credit for attendance. Then note the extra walk-ins who have decided to join.
		\item Then you may begin helping students.
		\item Refer people who walk in late who want to join the workshop to the next workshop, or other resources for help.
	\end{enumerate} 
	\item As the semester progresses we expect to see a spread in students level in the course. A group can be as small as two (but no greater than six) people. 
	\item If you come across an individual student in your workshop who is only a module or so behind/ahead the rest of the group you can offer them three options: 
	\begin{enumerate}
		\item Either work alone on the module they are on, 
		\item jump (if you feel they are capable of making this)/fallback to the module everyone is working on or leave the workshop, or
		\item e-mail the head TA so we can set up a group of other students in the same predicament.
	\end{enumerate}
\end{itemize}

\setlength{\headheight}{15pt}
\end{document}