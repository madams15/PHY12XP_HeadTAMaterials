\documentclass[12pt]{article}

\linespread{1}

\usepackage[english]{babel}
\usepackage{color}
\usepackage{enumerate}
\usepackage{fancyhdr}
\usepackage{float}
\usepackage[T1]{fontenc}
\usepackage{fullpage}
\usepackage[textheight = 675pt]{geometry}
\usepackage[pdftex]{graphicx}
\usepackage{hyperref} \hypersetup{colorlinks=true, urlcolor=blue}
\usepackage{ifpdf}
\usepackage[options]{pdfpages}
\usepackage{wrapfigure}
\usepackage{yfonts}

\pagestyle{fancy}
\setlength{\headheight}{10pt}
\lhead{M. Adams, A. Bodek *}
\chead{\emph{University of Rochester}}
\rhead{PHY122P, Fall 2015}}
\lfoot{* e-mail: \href{mailto:madams@pas.rochester.edu}{madams@pas.rochester.edu}, \newline \href{mailto:bodek@pas.rochester.edu}{bodek@pas.rochester.edu} }

\makeatletter
\renewcommand*{\p@section}{\S\,}
\renewcommand*{\p@subsection}{\S\,}
\renewcommand*{\p@subsubsection}{\S\,}
\makeatother

\begin{document}
\hspace{1pt}

\begin{center}
\textbf{\Large Grader}
\end{center}

In Mastery/Self-Paced Physics courses, grading is a significant component to the structure of the course. As a Grader, it is your job to objectively critique the work of a student on their module exam in about 7-8 minutes. It is also your responsibility to record their grades, prepare the physical exam to be taken, and manage the exam materials while on shift. This semester TAs who are grading can trade places with the TA Workshop Leader, in order to have breaks, and get a better understanding of what issues students are facing.

\begin{itemize}
	\item Grading take place in B\&L 208B (one of the adjoining rooms to B\&L 208). You will be informed on where to get keys at a later time. With these keys comes a set of smaller keys to the necessary materials for grading.
	\item Grading only take place during the scheduled workshops. There are 14 workshops running every week (M, T, W, R, F), except holidays (see Syllabus on Blackboard).
	\begin{itemize}
		\item Monday-Thursday: 14:00-16:40, 16:40-19:20, and 19:20-22:00.
		\item Friday: 14:00-16:40, 16:40-19:20 (no night workshop).
		\item No grading takes place during the Friday workshops.
	\end{itemize}
	\item Textbooks will be provided in B\&L 208.
	\item It is recommended that you familiarize yourself with the exam module solutions (to speed up grading) and the workshop modules/textbook (so on the Feedback Form you can tell them to go look at a particular problem/section).
	\item There are four things to consider while grading a module exam:
		\begin{enumerate}
			\item Whether a student understood the questions/concept in question,
			\item they knew the correct physics/equations to use,
			\item they were able to do the math/algebra correctly,
			\item and finally whether they ended up with the correct answer.
		\end{enumerate}
	\item As a rule of thumb if they understood the question, used the correct physics and were able to do the math that should account for 90\% of the grade (passing). The remaining 10\% is for getting it correct. A more structured rubric will be supplied a little later.
	\item Now for some math: if there is only one grader exam times are spaced ten minutes apart starting at 14:00 every day of the week; the last exam starts at 9:10pm. Thus there is a total of 8x6-5=43 exam slots per day. If there are two graders, exam times are spaced every 5 minutes, thus there is a total of 8x12-10=86 exam slots per day.
	\item Workflow of grading:
\setlength{\headheight}{15pt}
		\begin{enumerate}
			\item Use the computers provided to update their status on the course grade sheet (``Grade Center'' > ``Full Grade Center'' in Blackboard).

			\item For each student there are columns ``Test \# R-ABCD'' and ``Test \#.'' The following pattern is repeated for all modules in the course (See Table 1).
				\begin{itemize}
					\item The first column is intended to reflect whether a student is ready to take a test, and if he has taken a test which one.
					\item The second column reflects his highest grade on that test.
				\end{itemize}
			\item Check the google spreadsheet if they are scheduled for that time.
			\item Give the student a new test (never give the student the same test for a module), mark it on Blackboard.
			\item The student should have nothing on them but a student ID. They can leave their belongings in 208, or in the small hallway before the grading room. You give them a pencil and scrap paper.
			\item While students are taking the exam, try to occasionally be aware of things that are going on in the grading room, like talking.
			\item When a student comes out to have their exam graded, give them the appropriate mark \& feedback, and staple their scrap paper to their exam. No materials should leave the grading room.
			\item Put their exam materials in their designated folder.
		\end{enumerate}
	\item Your feedback on the exam will come in two forms:
		\begin{enumerate}
			\item A verbal section,
			\item and a written Feedback Form.
		\end{enumerate}
	\item On having a discussion with a student about their exam:
		\begin{itemize}
			\item During your verbal feedback you can show them where they went wrong but do not spend an excessive amount of time pointing out the mistake.
			\item It is important that this entire interaction, including grading, take no more than seven or eight minutes so you have two minutes to tend to the next student coming for the exam.
		\end{itemize}
	\item On \textbf{Feedback Forms}:
	\begin{itemize}
		\item These forms are provided after a Prescreen/Exam. On these you should fill out the name of the student in question, the date, and what Prescreen/Exam Module the comments pertain to.
		\item Comments can involve what mistakes they made, and what your recommendation is for them.
		\item If you are grading you give out Feedback Forms if the student passed the exam, and also if they failed the exam. If they pass, simply put the student's passing mark on it.
		\item The information on this Feedback Form is to be geared for two purposes. The first is for the student who wishes to correct their mistake on their own. The second is, to leave enough of a clue as to the issue so that a workshop leader can assist them. This Feedback Form is an important aspect of the feedback loop between the student and the grader.
		\item Feedback Form you can provide some more details, however don't give away the details of the question/solution. 
	\end{itemize}
\end{itemize}
\setlength{\headheight}{15pt}

\begin{table}[h]
\centering
\label{table:table}
\begin{tabular}{|p{3cm}|p{3cm}|p{8.5cm}|}
\hline
\textbf{Symbol} & \textbf{Best Grade} & \textbf{Explanation} \\                                                                                               \hline
R & 0 & R = Student has passed the Prescreen for Quiz \#1. \\ \hline
R-S & 0 & Student has passed the Prescreen interview for Quiz \#1 (R), and was scheduled (S) to take the quiz. \\ \hline
R-SAg & 0 & g = Student is taking quiz \#1A, g is removed when student gives the test to the grader. \\ \hline
R-SA & 70 & g is removed after test \#1A is graded. \\ \hline
R-SN & 0 & Student did not show (N) for first appointment to take Quiz \#1, which results in a zero, and still counts as a first attempt. \\ \hline
R-SA-S & 70 & Student was schedule (S) to take Quiz \#1 a second time. \\
R-SA-SBg & 70 & Student is taking quiz \#1B. \\ \hline
R-SA-SB & 100 & g is removed after test \#1B is taken, and the student has passed with more than 80\%. \\ \hline
R-SA-SB & 80 & g is removed after test \#1B is taken with grade of 80. Since student failed two times, they need to pass a more rigorous Prescreen. \\ \hline
R-SA-SB-R & 80 & Student has been prescreened a second time (R2). \\ \hline
R-SA-SB-R-S & 80 & Student has been prescreened a second time (R2), and scheduled to take the test a third time. \\
\hline
\end{tabular}}
\caption{The first two columns illustrate what information will be input into the column labeled ``Test \# R-ABCD'' on Blackboard. Note that 'g' is used to ensure that no test leaves the grading area. The letter 'S' is to prevent students from scheduling Module Exam \#2 before passing Module Exam \#1, or scheduling two slots for the exam.}
\end{table}

\end{document}